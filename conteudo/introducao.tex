\subsection{Contexto}
	

	O Enturma foi desenvolvido por alunos do curso de Engenharia de Software da Universidade de Brasília, com o objetivo de apresentar para a população brasileira uma análise da educação brasileira ao longo do tempo. Esta análise é feita com base em dados recolhidos dos Dados Abertos do Governo Federal, e a análise engloba questões como Evasão, Rendimento, nota do IDEB e Distorção.

	O usuário, utilizando o Enturma, pode verificar a qualidade do ensino em seu estado, podendo ainda realizar uma comparação entre dois estados, observando detalhes importantes sobre a situação da educação brasileira nos ultimos anos.

	A necessidade da criação de um sistema como o Enturma surgiu com a criação de uma lei, em 2008 na qual foi decretado que, a partir daquele ano, nenhum aluno poderia ser reprovado no primeiro ano de ensino, e ainda uma recomendação às escolas para evitarem reprovar alunos no segundo ano de ensino.

	Dada esta lei, surgiu a necessidade de verificar o impacto da mesma na educação brasileira. O aluno que não reprova, mesmo que ainda analfabeto, no primeiro e segundo ano de ensino, consegue ser aprovado tranquilamente no terceiro ou quarto ano? A partir deste questionamento, a equipe de Alunos da Universidade de Brasília desenvolveu o Enturma com o objetivo de solucionar este problema.

\subsection{Formulação do problema}
	
	O sistema Enturma possui grande valor para a população brasileira que possui interesse em obter dados referentes a educação brasileira. Porém, podemos garantir que qualquer usuário poderá acessar o sistema e usufruir de suas funcionalidades de uma forma simples, prática ou até mesmo, divertida? 

	A avaliação da Usabilidade do sistema Enturma é de extrema importância para que correções possam ser realizadas com o objetivo de garantir uma boa experiência de uso dos usuários finais, garantindo assim, o acesso adequado para todos os tipos de Usuários.

\subsection{Objetivos}

	O objetivo deste trabalho é avaliar a Usabilidade do Sistema Enturma, apontar falhas e corrigir o máximo possível de problemas de Acessibilidade ou Usabilidade, para garantir a satisfação do usuário final, seja ele quem for. O Sistema Enturma é um sistema \textit{Multiplataforma}, possuindo uma versão Web, uma versão em Aplicativo para Android e outra em Aplicativo para IOS.

	Garantir a avaliação e correção das três plataformas é inviável levando em consideração o período de tempo destinado para esta atividade, dessa forma, o foco deste trabalho será no sistema Web, desenvolvido em Ruby on Rails.



\subsection{Justificativas}

	A experiência de uso do após utilizar um sistema de Software é extremamente importante para o sucesso do Software no mercado. Garantir que o Usuário manuseie o Software de forma simples e eficiente é a grande dificuldade do estudo da Interação Humano-Computador (IHC).
	O Sistema Enturma possui grande importância para a população brasileira, dessa forma, a Acessibilidade do sistema e a garantia de uma boa Usabilidade se torna essencial, já que o sistema tem como objetivo, alcançar o máximo de cidadãos brasileiros possível.

	Como o objetivo deste projeto é garantir qualidade na Usabilidade do Sistema, torna-se necessário a utilização de Avaliações de Usabilidade e Acessibilidade do Sistema Enturma, facilitando a obtenção do maior número de defeitos de Acessibilidade e Usabilidade possíveis, para que possam ser corrigidos, evoluindo a Usabilidade do Enturma.
	
	As Avaliações serão realizadas com base em testes de usabilidade utilizando as 10 (dez) heuristicas de Nielsen \cite{usabilidade_web}, pois, com isso, garantimos uma análise rápida, simples e que viabiliza a obtenção de dados importantíssimos para a Análise da Usabilidade do Enturma.


