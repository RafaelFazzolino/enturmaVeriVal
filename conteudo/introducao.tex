\subsection{Contexto}
	

	O Enturma foi desenvolvido por alunos do curso de Engenharia de Software da Universidade de Brasília, com o objetivo de apresentar para a população brasileira uma análise da educação brasileira ao longo do tempo. Esta análise é feita com base em dados recolhidos dos Dados Abertos do Governo Federal, e a análise engloba questões como Evasão, Rendimento, nota do IDEB e Distorção.

	O usuário, utilizando o Enturma, pode verificar a qualidade do ensino em seu estado, podendo ainda realizar uma comparação entre dois estados, observando detalhes importantes sobre a situação da educação brasileira nos ultimos anos.

	A necessidade da criação de um sistema como o Enturma surgiu com a criação de uma lei, em 2008 na qual foi decretado que, a partir daquele ano, nenhum aluno poderia ser reprovado no primeiro ano de ensino, e ainda uma recomendação às escolas para evitarem reprovar alunos no segundo ano de ensino.

	Dada esta lei, surgiu a necessidade de verificar o impacto da mesma na educação brasileira. O aluno que não reprova, mesmo que ainda analfabeto, no primeiro e segundo ano de ensino, consegue ser aprovado tranquilamente no terceiro ou quarto ano? A partir deste questionamento, a equipe de Alunos da Universidade de Brasília desenvolveu o Enturma com o objetivo de solucionar este problema.

\subsection{Formulação do problema}
	
	O sistema Enturma possui grande valor para a população brasileira que possui interesse em obter dados referentes a educação brasileira. Porem, podemos garantir que qualquer usuário poderá acessar o sistema e usufruir de suas funcionalidades de uma forma simples, prática ou até mesmo, divertida? 

	A avaliação da Usabilidade do sistema Enturma é de extrema importância para que correções possam ser realizadas com o objetivo de garantir uma boa experiência de uso dos usuários finais e garantindo, assim, o acesso adequado para todos os tipos de Usuários.

\subsection{Objetivos}

	O objetivo deste trabalho é avaliar a Usabilidade do Sistema Enturma, apontar falhas e corrigir o máximo possível de problemas de Acessibilidade ou Usabilidade para garantir a satisfação do usuário final, seja ele quem for. O sistema Enturma é um sistema \textit{Multiplataforma}, possuindo uma versão Web, uma versão em Aplicativo para Android e outra em Aplicativo para IOS.

	Garantir a avaliação e correção das três plataformas é inviável levando em consideração o período de tempo destinao para esta atividade, dessa forma, o foco deste trabalho será no sistema Web, desenvolvido em Ruby on Rails.

\subsection{Justificativas}

	
	Com objetivo de melhorar a usabilidade e promover uma melhor experiencia ao usuario ao utilizar o software, serão feitos testes de usabilidade utilizando as 10 heuristicas de Nielsen, pois garante uma análise rápida de ser aplicada e viabiliza a obtenção de dados importantíssimos para a Avaliação da Usabilidade do Enturma.
	Também será elaborado um questionario que servirá como auxilio para mudanças posteriores.
	Realizar um teste intuitivo com o usuario com um roteiro previamente elaborado que será passado ao usuario,
	o usuario será filmado utilizando o software para avaliação posterior.
	Se possivel alterar a usabilidade do software para utilização de deficientes visuais.
	Aplicar modelo de acessibilidade em Governo eletronico (eMAG) consiste em um conjunto de recomendações a ser considerado para que o processo de acessibilidade dos sítios e portais do governo brasileiro seja conduzido de forma padronizada e de fácil implementação.
	(MELHORAR JUSTIFICATIVAS)



