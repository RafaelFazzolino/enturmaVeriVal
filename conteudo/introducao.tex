\subsection{Contexto}

	O Enturma trata a cerca do problema de falta de visibilidade em relação a educação no Brasil e com base nisso surgiu a motivaçao para tal problema.
	O Enturma é um software com intuito de auxiliar o cidadão brasileiro a acompanhar e obter informações sobre o rengimento das turmas do ensino fundamental no Brasil, por indices disponibilizados pelo INEP, para assim o usuario ver o desempenho de uma turma especifica, em um determinado período, com dados a respeito da turma a ser acompanhada.
	Após o Enturma esta concluido, foi percebido que o mesmo não segue alguns padrões de usabilidades o que deixa a questão de interface com problemas (continuar descrevendo os problemas)

\subsection{Formulação do problema}

	[Responder à questão: QUAIS AS QUESTÕES A SEREM RESOLVIDAS?
Detalhar o problema citado no Contexto e apresentar as questões específicas que a presente pesquisa pretende responder ou resolver. São as questões a serem solucionadas.]

\subsection{Objetivos}

	[ Responder à questão: O QUE FAZER?
Apresentar os objetivos como Objetivo Geral e Objetivos específicos.

Objetivo Geral: o que se pretende atingir / alcançar com a pesquisa

Objetivos específicos: São etapas do trabalho para se alcançar / atingir o objetivo geral. Utilizar verbos no infinitivo para descrever tais objetivos específicos.

Exploratórios (conhecer, identificar, levantar, descobrir)

Descritivos (caracterizar, descrever, traçar, determinar, definir)

Explicativos (analisar, avaliar, verificar, explicar, validar)   ]

\subsection{Justificativas}

	[Basicamente, essa seção deve responder à seguinte questão: POR QUE FAZER?
As justificativas consistem em uma descrição e argumentações sobre as razões e motivações da escolha do tema de projeto em questão, de maneira a esclarecer as razões pelas quais o presente projeto é importante.
Essa descrição/argumentação deve indicar:

A importância do tema a ser investigado.

As possíveis contribuições do projeto.

Relação do tema com outras pesquisas.