\subsection{Contexto}
	

	O Enturma foi desenvolvido por alunos do curso de Engenharia de Software da Universidade de Brasília, com o objetivo de apresentar para a população brasileira uma análise da educação brasileira ao longo do tempo. Esta análise é feita com base em dados recolhidos dos Dados Abertos do Governo Federal, e a análise engloba questões como Evasão, Rendimento, nota do IDEB e Distorção.

	O usuário, utilizando o Enturma, pode verificar a qualidade do ensino em seu estado, podendo ainda realizar uma comparação entre dois estados, observando detalhes importantes sobre a situação da educação brasileira nos ultimos anos.

	A necessidade da criação de um sistema como o Enturma surgiu com a criação de uma lei, em 2008 na qual foi decretado que, a partir daquele ano, nenhum aluno poderia ser reprovado no primeiro ano de ensino, e ainda uma recomendação às escolas para evitarem reprovar alunos no segundo ano de ensino.

	Dada esta lei, surgiu a necessidade de verificar o impacto da mesma na educação brasileira. O aluno que não reprova, mesmo que ainda analfabeto, no primeiro e segundo ano de ensino, consegue ser aprovado tranquilamente no terceiro ou quarto ano? A partir deste questionamento, a equipe de Alunos da Universidade de Brasília desenvolveu o Enturma com o objetivo de solucionar este problema.

\subsection{Formulação do problema}
	
	O sistema Enturma possui grande valor para a população brasileira que possui interesse em obter dados referentes a educação brasileira. Porem, podemos garantir que qualquer usuário poderá acessar o sistema e usufruir de suas funcionalidades de uma forma simples, prática ou até mesmo divertida? 

	A avaliação da Usabilidade do sistema Enturma é de extrema importância para que correções possam ser realizadas com o objetivo de garantir uma boa experiência de uso dos usuários finais.

\subsection{Objetivos}

	O objetivo deste trabalho é avaliar a Usabilidade do Sistema Enturma, apontar falhas e corrigir o máximo possível para garantir a satisfação do usuário final, seja ele quem for. O sistema Enturma é um sistema \textit{Multiplataforma}, possuindo uma versão Web, uma versão em Aplicativo para Android e outra para IOS.

	Garantir a avaliação e correção das três plataformas é inviável, dessa forma, o foco deste trabalho será no sistema Web, desenvolvido em Ruby on Rails.

\subsection{Justificativas}

	[Basicamente, essa seção deve responder à seguinte questão: POR QUE FAZER?
As justificativas consistem em uma descrição e argumentações sobre as razões e motivações da escolha do tema de projeto em questão, de maneira a esclarecer as razões pelas quais o presente projeto é importante.
Essa descrição/argumentação deve indicar:

A importância do tema a ser investigado.

As possíveis contribuições do projeto.

Relação do tema com outras pesquisas.