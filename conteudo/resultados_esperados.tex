
	O projeto de Avaliação da Usabilidade e Acessibilidade do Sistema Enturma possui 2 (duas) fazes, as quais têm como objetivo Evoluir a Usabilidade do Sistema Enturma. A primeira fase do projeto, tem como objetivo principal, encontrar o maior número de defeitos de Usabilidade e Acessibilidade possível. Obtendo, assim, o máximo de dados referentes a qualidade da Usabilidade do Sistema, facilitando um planejamento de manutenção do mesmo.

	A segunda fase do projeto, tem como objetivo realizar manutenção Corretiva/Evolutiva no Enturma, utilizando como insumo os dados obtidos na primeira fase do projeto. Espera-se corrigir o máximo de problemas possível, para que a experiência de uso dos Usuários seja qualificada.

	Com o fim da segunda fase do projeto, possuiremos todos os dados necessários para a realização de uma análise da Evolução da Usabilidade e Acessibilidade do Sistema. Com esta análise, espera-se obter um resultado que apresente uma melhora significativa na Usabilidade e Acessibilidade do sistema Enturma Web.