\subsection{Técnicas de Questionario}
	Técnicas de questionario são aplicadas em testes de usabilidade em que há participação do usuario com o objetivo de perguntar ao usuario sobre a interface do software com o objetivo de descobrir se o software esta suprindo as necessidades do usuario.

	Na maioria das vezes um modelo de questionário apóia-se nas experiências e
	heurísticas de seus elaboradores. Quando é utilizado em pesquisas reais ou simuladas, o
	modelo depara-se com circunstâncias e necessidades não previstas inicialmente, o que
	determinará os refinamentos e ajustes, que, aplicados sucessivamente, permitirão a
	evolução das questões (NIELSEN; MACK, 1994). 
	
	As técnicas são úteis para se obter detalhes que do ponto de vista nós desenvolvedores não estamos acostumados, e são utilizadas para obter informações relativas as necessidades dos usuarios e revelar possiveis problemas que nós desenvolvedores normalmente nao veriamos.
	
	Serão realizados questionarios afim de se obter dados quantativos, o questionario também tem como vantagem atingir um numero maior de usuarios o que facilita ainda mais a coleta de dados relevantes a respeito do desing do software.
	
	


\subsection{Planejamento da avaliação de usabilidade e questionario de satisfação dos usuários}
	A avaliação da usabilidade será realizada com a utilização do software enturma, facilitando o registro de problemas encontrados durante a utilização do sistema pelos usuários da aplicação.
	A proposta do questionario a ser aplicado aos usuários após a realização da avaliação de usabilidade com base no WAMMI e no QUIS. segue a proposta de questionario:
	