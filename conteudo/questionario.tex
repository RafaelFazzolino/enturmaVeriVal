\subsection{Técnicas de Questionario}
	Técnicas de questionario são aplicadas em testes de usabilidade em que há participação do usuario com o objetivo de perguntar ao usuario sobre a interface do software com o objetivo de descobrir se o software esta suprindo as necessidades do usuario.

	Na maioria das vezes um modelo de questionário apóia-se nas experiências e
	heurísticas de seus elaboradores. Quando é utilizado em pesquisas reais ou simuladas, o
	modelo depara-se com circunstâncias e necessidades não previstas inicialmente, o que
	determinará os refinamentos e ajustes, que, aplicados sucessivamente, permitirão a
	evolução das questões (NIELSEN; MACK, 1994). 
	
	As técnicas são úteis para se obter detalhes que do ponto de vista nós desenvolvedores não estamos acostumados, e são utilizadas para obter informações relativas as necessidades dos usuarios e revelar possiveis problemas que nós desenvolvedores normalmente nao veriamos.
	
	Serão realizados questionarios afim de se obter dados quantativos, o questionario também tem como vantagem atingir um numero maior de usuarios o que facilita ainda mais a coleta de dados relevantes a respeito do desing do software.
	
	


\subsection{Planejamento da avaliação de usabilidade e questionario de satisfação dos usuários}
	A avaliação da usabilidade será realizada com a utilização do software enturma, facilitando o registro de problemas encontrados durante a utilização do sistema pelos usuários da aplicação.
	A proposta do questionario a ser aplicado aos usuários após a realização da avaliação de usabilidade com base no WAMMI (Website Analysis and MeasurMent Inventory) e no QUIS(Questionnaire for User Interactional Satisfaction).


	O QUIS é uma ferramenta que foi desenvolvida por pesquisadores do Human Computer Interaction Laboratory (HCIL) da University of Maryland, para medir a satisfação do usuario focando em objetivos especificos da interface humano-computador.
	As questões são apresentadas na forma de afirmações
utilizando as escalas de diferencial semântico, que
baseiam-se em explorar uma faixa de atitudes bipolares
representada por um par de adjetivos. As questões são
respondidas em uma escala que varia de O a 9, onde o zero
representa um adjetivo negativo e os demais representam
adjetivos positivos. Por ser um questionário geral utilizado
para uma ampla variedade de produtos, também inclui a
opção N/A (não-aplicável). A Figura 1 mostra um exemplo
de uma questão com escalas de satisfação específica. ( Filardi; Traina, 2008)

\begin{table}
{Exemplo de uma questão de QUIS utilizando escala de diferencial semantico}
\centering
\begin{tabular}{|c|c|c|c|c|c|c|c|c|c|c|c|c|c|c|} 

\hline
   QUIS &  & 1 & 2 & 3 & 4 & 5 & 6 & 7 & 8 & 9 & N/A &  \\
\hline
Mensagens que aparecem na tela & Confusa &  &  &  &  &  &  &  &  &  &  & clara \\ 

\hline
\end{tabular}
\caption {Fonte:Filardi; Traina, 2008}

\end{table}

	O WAMMI é um serviço exclusivo para avaliação de
Websites on-Iine, com o propósito de ajudar os proprietários
do site a cumprir suas metas corporativas através da
medição e monitoramento das reações do usuário sobre
suafacilidade de uso. Através de um botão colocado no site,
é disponibilizado um questionário com a estrutura de um
formulário para ser preenchido. Os dados do questionário
são armazenados e analisados a partir de uma base de dados
padronizada com escores normalizados. São utilizados para
avaliar os seguintes aspectos: atratividade, controle,
eficiência, utilidade, aprendizagem e usabilidade global.
	
	O WAMMI tem como objetivo: 

• medir a satisfação do usuário sobre o site baseado na
reação do usuário;

• gerar dados objetivos de gestão em um formato fácil de
entender;

• prover uma base para mudanças do Website e melhorias
de design;

• comparar seu site em relação aos demais em termos de
satisfação do usuário; 

• acompanhar o desempenho do Website para verificar se
as metas estão sendo cumpridas.

Fonte:Filardi; Traina, 2008