
O escopo inicial do projeto englobava desde a Avaliação e correção dos problemas voltados a Usabilidade do sistema, ou seja, a experiência de uso do Usuário, até a Avaliação e correção dos problemas voltados a Acessibilidade do sistema.

Para avaliar e corrigir os problemas que se enquadram na primeira etapa, que se refere a Usabilidade do sistema, foram utilizados questionários que obtem os dados sobre a experiência de uso dos usuários. Foram utilizadas pessoas de diferentes idades e classes sociais para que possamos realizar uma amostragem de qualidade.

Para avaliar e corrigir os problemas que se enquadram na segunda etapa, que se refere a Acessibilidade do sistema, planejamos a utilização do sistema ASES, que é um \textit{software} desenvolvido pelo Governo Federal e está disponível clicando \href{http://www.governoeletronico.gov.br/acoes-e-projetos/e-MAG/ases-avaliador-e-simulador-de-acessibilidade-sitios}{\textbf{AQUI}}. O sistema ASES foi criado com o objetivo de avaliar sites observando critérios que são descritos no conjunto de padrões e boas práticas chamado \textit{e-Mag}, também desenvolvido pelo Governo Federal. O e-Mag leva em consideração detalhes que ajudam a garantir que qualquer cidadão poderá acessar e navegar com qualidade dentro do sistema web, ou seja, é extremamente importante para o nosso contexto, e agregaria bastante valor com a sua utilização.

Porem, durante a utilização do sistema ASES, foi observado que os resultados apresentados pelo sistema são incosistentes e incoerentes. Não agregando valor algum ao projeto de Avaliação do sistema Enturma. Após inúmeras pesquisas e análises foi observado que o sistema ASES é um sistema com inúmeras falhas e que não está sendo mantido, é um sistema descontinuado, esquecido pelo Governo Federal.

Utilizar um sistema falho para avaliação do Enturma pode gerar inúmeros problemas e imperfeições, e por esse motivo decidimos retirá-lo do escopo do projeto, buscando novas alternativas que possam substituir a responsabilidade alocada para o sistema ASES.

\subsection{Alternativa Verificada} % (fold)
\label{sub:alternativa_verificada}

	Com o objetivo de substituir o sistema ASES durante o projeto de avaliação do sistema Enturma, buscamos inúmeras fontes que realizem a mesma funcionalidade do ASES, porém com resultados consistentes e úteis para o projeto. Durante esta busca, encontramos outro \textit{software} desenvolvido pelo Governo Federal, chamado \textit{daSilva}.

	O projeto \textit{daSilva} é um sistema bem atual, simples, eficiente e que engloba, ainda, todas as características desejadas pelo projeto. Ou seja, seria uma alternativa perfeita para substituir o sistema ASES durante o projeto de avaliação do sistema Enturma. Porém, como nada é perfeito, o sistema \textit{daSilva} se encontra fora do ar e só possui versão web, que estava disponível em \href{http://www.dasilva.org.br/}{\textbf{AQUI}}. 

	Dessa forma, a avaliação da Acessibilidade do sistema Enturma foi retirada do escopo por falta de ferramentas que apoiem o a avaliação da mesma.
% subsection alternativa_verificada (end)