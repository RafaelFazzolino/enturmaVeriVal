\chapter*{Anexos}

\begin{itemize}

	\item \textbf{Termo de Conscentimento}

		Os dados referentes aos participantes dos questionários serão mantidos em sigilo e apenas terão acesso a esses dados os pesquisadores participantes da pesquisa.	
		Em qualquer momento você terá acesso aos profissionais responsáveis pela pesquisa para esclarecimento de possiveis duvidas.

		Não haverá qualquer tipo de desembolso pessoais para os participantes em qualquer fase desta pesquisa, e não haverá compensação financeira relacionada a sua participação.

		As informações coletadas serão utilizadas somente para fins cientifícos e acadêmicos, sendo divulgados apenas em trabalhos ou artigos acadêmicos, preservando assim o anonimato dos participantes.
			
		Após ter lido entendido o texto acima e ter exclarecido as duvidas adicionais sobre o estudo. \\


		\textbf{Consentimento Pós–Informação} \\

		Eu,\_\_\_\_\_\_\_\_\_\_\_\_\_\_\_\_\_\_\_\_\_\_\_\_\_\_\_\_\_\_\_\_\_\_\_\_\_\_, fui informado
		sobre o que o pesquisador deseja fazer e porque precisa da minha colaboração.
		 Por isso, eu concordo em participar desta pesquisa, sabendo que não vou obter nenhum tipo de ganho
		e que posso desistir da pesquisa quando assim desejar. Este documento é emitido em duas vias que serão ambas
		assinadas por mim e pelo pesquisador, ficando uma via com cada um de nós.


	\item \textbf{Questionário Perfil do usuario}


		\textbf{1.0 Sexo:}
			\\
			  ( ) Feminino \\
			  ( ) Masculino\\
			\\
		\textbf{2.0 Faixa de Idade:}
			\\
			  ( ) Até 17 anos\\
			  ( ) 18 à 24 anos\\
			  ( ) 25 à 44 anos\\
			  ( ) 45 à 64 anos\\
			  ( ) Acima de 65 anos\\
			\\
		\textbf{3.0 Grau de Instrução:}
			\\
			  ( ) Ensino médio\\
			  ( ) Superior completo\\
			  ( ) Superior incompleto\\
			\\

		\begin{table}[H]
			\centering
			\begin{tabular}{|l|l|l|l|l|}
				\hline
				\multicolumn{5}{|c|}{\textbf{Experiência com Informática}}                                                                                                                                                                                                                                                                           \\ \hline
				Frequência de uso                                                               & {[}  {]}Diária                                                     & {[}  {]}Semanal           & {[}  {]}Eventual                                                    & {[}  {]}Nunca                                                               \\ \hline
				\begin{tabular}[c]{@{}l@{}}Possui computador a \\ quanto tempo?\end{tabular}    & {[}  {]} Até 1 ano                                                 & {[}  {]} Entre 1 e 2 anos & {[}  {]} Entre 2 e 5 anos.                                          & {[}  {]} Mais de 5 anos                                                     \\ \hline
				\begin{tabular}[c]{@{}l@{}}Importância da Internet \\ em sua vida.\end{tabular} & \begin{tabular}[c]{@{}l@{}}{[}  {]}Sem \\ importância\end{tabular} & {[}  {]} Importante       & \begin{tabular}[c]{@{}l@{}}{[}  {]} Muito\\ Importante\end{tabular} & \begin{tabular}[c]{@{}l@{}}{[}  {]} Extremamente\\ Importante.\end{tabular} \\ \hline
			\end{tabular}
			\caption{Experiência em Informática}
			\label{tab:experienciaInformatica}
		\end{table}



	Para uma melhor visualização dos problemas relacionados ao Projeto Enturma, será aplicado um questionário baseado no modelo \textit{QUIS} para obter dados referentes a problemas de interface. Afim de realizar melhorias na interface do sistema, com objetivo de tornar o serviço mais acessivel e interessante para a população brasileira.

	A avaliação é feita da seguinte maneira:

		A tabela é classificada como Confusa ou clara variando uma pontuação de 1 a 9, sendo 1 confusa, ou seja  obteve-se dificuldade e 9 sendo clara, não se teve dificuldade em encontrar a informação.

		\begin{table}[H]
	{Quis para avaliação do enturma}
	\centering
	\begin{tabular}{|c|c|c|c|c|c|c|c|c|c|c|c|c|c|c|} 

		\hline
		   \textbf{Visualizar ranking} &  & 1 & 2 & 3 & 4 & 5 & 6 & 7 & 8 & 9 & N/A &  \\
		\hline
		   Acesso ao site Enturma & Confusa &  &  &  &  &  &  &  &  &  &  & clara \\ 
		\hline
		 Acesso a seção de serviços  & Confusa &  &  &  &  &  &  &  &  &  &  & clara \\ 
		\hline
		 Escolha de uma Turma  & Confusa &  &  &  &  &  &  &  &  &  &  & clara \\ 
		\hline
		 Visualização dos dados do ranking & Confusa &  &  &  &  &  &  &  &  &  &  & clara \\ 
		\hline
		   \textbf{Acompanhamento da turma} &  & 1 & 2 & 3 & 4 & 5 & 6 & 7 & 8 & 9 & N/A &  \\
		\hline
		 Visualizar relatório de turmas & Confusa &  &  &  &  &  &  &  &  &  &  & clara \\ 
		\hline
		  Achar turma desejada pelos filtros & Confusa &  &  &  &  &  &  &  &  &  &  & clara \\ 
		\hline
		  Visualização de todos dados obtidos & Confusa &  &  &  &  &  &  &  &  &  &  & clara \\ 
		\hline
		  Limpar a pesquisa e a realizar uma nova pesquisa & Confusa &  &  &  &  &  &  &  &  &  &  & clara \\ \hline
		\textbf{Comparar Turmas} &  & 1 & 2 & 3 & 4 & 5 & 6 & 7 & 8 & 9 & N/A &  \\
		\hline  
		  Selecionar Comparar Turmas & Confusa &  &  &  &  &  &  &  &  &  &  & clara \\ 
		\hline
		  Selecionar duas turmas com seus filtros & Confusa &  &  &  &  &  &  &  &  &  &  & clara \\ 
		\hline
		 Visualização de todos dados obtidos & Confusa &  &  &  &  &  &  &  &  &  &  & clara \\ 
		\hline
		Limpar a pesquisa e a realizar uma nova pesquisa & Confusa &  &  &  &  &  &  &  &  &  &  & clara \\ 
		\hline
		\textbf{Interface} &  & 1 & 2 & 3 & 4 & 5 & 6 & 7 & 8 & 9 & N/A &  \\
		\hline  
		  Layout do site & confusa &  &  &  &  &  &  &  &  &  &  & clara \\ 
		\hline
		  Aspecto visual do site & confusa &  &  &  &  &  &  &  &  &  &  & clara \\ 
		\hline
		  Os elementos de informação são dispostos nas\\ páginas de forma organizada e racional?  & Confusa &  &  &  &  &  &  &  &  &  &  & clara \\ 
		\hline
		Limpar a pesquisa e a realizar uma nova pesquisa & Confusa &  &  &  &  &  &  &  &  &  &  & clara \\ 
		\hline

	\end{tabular}
	\caption {Quis enturma}

\end{table}

\end{itemize}