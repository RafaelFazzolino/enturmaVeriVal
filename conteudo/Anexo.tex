\chapter*{Anexos}

\begin{itemize}

	\item \textbf{Termo de Conscentimento}

		Todos as informações pessoais obtidas durante a aplicação deste questionário serão mantidas em sigilo e só serão utilizados para obter informação sobre a aceitabilidade do sistema Enturma por parte dos usuários. \\
		Eu,\_\_\_\_\_\_\_\_\_\_\_\_\_\_\_\_\_\_\_\_\_\_\_\_\_\_\_\_\_\_\_\_\_\_\_\_\_\_, estou disposto a informar meus dados pessoais, assim como utilizar o sistema Enturma com o objetivo de obter e repassar a Experiência de Uso do mesmo.



	\item \textbf{Questionário Perfil do usuario}


		\textbf{1.0 Sexo:}
			\\
			  ( ) Feminino \\
			  ( ) Masculino\\
			\\
		\textbf{2.0 Faixa de Idade:}
			\\
			  ( ) Até 17 anos\\
			  ( ) 18 à 24 anos\\
			  ( ) 25 à 44 anos\\
			  ( ) 45 à 64 anos\\
			  ( ) Acima de 65 anos\\
			\\
		\textbf{3.0 Grau de Instrução:}
			\\
			  ( ) Ensino médio\\
			  ( ) Superior completo\\
			  ( ) Superior incompleto\\
			\\

		\begin{table}[H]
			\centering
			\begin{tabular}{|l|l|l|l|l|}
				\hline
				\multicolumn{5}{|c|}{\textbf{Experiência com Informática}}                                                                                                                                                                                                                                                                           \\ \hline
				Frequência de uso                                                               & {[}  {]}Diária                                                     & {[}  {]}Semanal           & {[}  {]}Eventual                                                    & {[}  {]}Nunca                                                               \\ \hline
				\begin{tabular}[c]{@{}l@{}}Possui computador a \\ quanto tempo?\end{tabular}    & {[}  {]} Até 1 ano                                                 & {[}  {]} Entre 1 e 2 anos & {[}  {]} Entre 2 e 5 anos.                                          & {[}  {]} Mais de 5 anos                                                     \\ \hline
				\begin{tabular}[c]{@{}l@{}}Importância da Internet \\ em sua vida.\end{tabular} & \begin{tabular}[c]{@{}l@{}}{[}  {]}Sem \\ importância\end{tabular} & {[}  {]} Importante       & \begin{tabular}[c]{@{}l@{}}{[}  {]} Muito\\ Importante\end{tabular} & \begin{tabular}[c]{@{}l@{}}{[}  {]} Extremamente\\ Importante.\end{tabular} \\ \hline
			\end{tabular}
			\caption{Experiência em Informática}
			\label{tab:experienciaInformatica}
		\end{table}



	Para uma melhor visualização dos problemas relacionados ao Projeto Enturma, será aplicado um questionário baseado no modelo \textit{QUIS} para obter dados referentes a problemas de interface. Afim de realizar melhorias na interface do sistema, com objetivo de tornar o serviço mais acessivel e interessante para a população brasileira.

	A avaliação é feita da seguinte maneira:

		A tabela é classificada como Confusa ou clara variando uma pontuação de 1 a 9, sendo 1 confusa, ou seja  obteve-se dificuldade e 9 sendo clara, não se teve dificuldade em encontrar a informação.

\begin{table}[h]
\centering
\caption{Quis}
\label{my-label}
\begin{tabular}{|l|l|l|l|l|l|l|l|l|l|l|}
\hline
\multicolumn{2}{|l|}{\textbf{Visualizar ranking}}                                                                                         & 1 & 2 & 3 & 4 & 5 & 6 & 7 & 9 &      \\ \hline Acesso ao site Enturma                                                                                                          & Confusa &   &   &   &   &   &   &   &   & Clara \\ \hline
Acesso a seção de serviços                                                                                                      & Confusa &   &   &   &   &   &   &   &   & Clara \\ \hline
Escolha de uma Turma                                                                                                            & Confusa &   &   &   &   &   &   &   &   & Clara \\ \hline
Visualização dos dados do ranking                                                                                               & Confusa &   &   &   &   &   &   &   &   & Clara \\ \hline
\multicolumn{11}{|l|}{\textbf{Acompanhamento da turma}}                                                                                                                           \\ \hline
Visualizar relatório de turmas                                                                                                  & Confusa &   &   &   &   &   &   &   &   & Clara \\ \hline
Achar turma desejada pelos filtros                                                                                              & Confusa &   &   &   &   &   &   &   &   & Clara \\ \hline
Visualização de todos dados obtidos                                                                                             & Confusa &   &   &   &   &   &   &   &   & Clara \\ \hline
Limpar a pesquisa e a realizar uma nova pesquisa                                                                                & Confusa &   &   &   &   &   &   &   &   & Clara \\ \hline
\multicolumn{11}{|l|}{\textbf{Comparar Turmas}}                                                                                                                                   \\ \hline
Selecionar Comparar Turmas                                                                                                      & Confusa &   &   &   &   &   &   &   &   & Clara \\ \hline
Selecionar duas turmas com seus filtros                                                                                         & Confusa &   &   &   &   &   &   &   &   & Clara \\ \hline
Visualização de todos dados obtidos                                                                                             & Confusa &   &   &   &   &   &   &   &   & Clara \\ \hline
Limpar a pesquisa e a realizar uma nova pesquisa                                                                                & Confusa &   &   &   &   &   &   &   &   & Clara \\ \hline
\multicolumn{11}{|l|}{\textbf{Interface}}                                                                                                                                         \\ \hline
Layout do site                                                                                                                  & Confusa &   &   &   &   &   &   &   &   & Clara \\ \hline
Aspecto visual do site                                                                                                          & Confusa &   &   &   &   &   &   &   &   & Clara \\ \hline
\begin{tabular}[c]{@{}l@{}}Os elementos de informação são dispostos\\  nas páginas de forma organizada e racional?\end{tabular} & Confusa &   &   &   &   &   &   &   &   & Clara \\ \hline
\end{tabular}
\end{table}

\end{itemize}