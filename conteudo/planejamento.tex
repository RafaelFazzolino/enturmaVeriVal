
Com o intuito de planejar a Verificação e Validação da Interface Gráfica que compõe o sistema Enturma, primeiramente precisamos compreender as funcionalidades principais do sistema, seu objetivo principal, seus objetivos secundários e os pontos mais críticos do sistema.

O sistema Enturma possui como objetivo, informar a população brasileira sobre o desempenho das turmas escolares no país, facilitando o conhecimento a cerca da distribuição de ensino por todo o país, apresentando com clareza as diferenças entre cada estado e seus tipos escolares (público, privado, estadual, municipal). 

Obter a informação a cerca de uma determinada turma é o principal objetivo do sistema, já que o usuário poderá saber a evolução de sua turma, podendo compará-la com outras turmas, de outros estados por exemplo. O \textit{framework do problema} pode ser observado na tabela \ref{tab:problema}.


\begin{table}[H]
	\centering
	\begin{tabular}{|l|l|}
		\hline
		\textbf{O problema de}                    & \begin{tabular}[c]{@{}l@{}}Falta de conhecimento a cerca do desempenho\\ das turmas escolares brasileiras.\end{tabular}                                                                                                                                                                         \\ \hline
		\textbf{Afeta}                            & Todos os brasileiros.                                                                                                                                                                                                                                                                           \\ \hline
		\textbf{Cujo impacto é}                   & \begin{tabular}[c]{@{}l@{}}O decaimento do rendimento dos alunos fica a margem \\ do conhecimento da população brasileira, estando essa \\ sem ter como acompanhar quando é que a educação \\ começa a oscilar para poder exigir investimento de \\ recursos por parte do governo.\end{tabular} \\ \hline
		\textbf{Benefícios de uma solução seriam} & \begin{tabular}[c]{@{}l@{}}Exigência por parte da população para a elaboração de \\ leis e medidas que focariam melhorias na educação em \\ fases críticas do ensino, as quais se apresentam \\ defasadas nos resultados das provas avaliativas.\end{tabular}                                   \\ \hline
	\end{tabular}
	\caption{Framework de Problema}
	\label{tab:problema}
\end{table}