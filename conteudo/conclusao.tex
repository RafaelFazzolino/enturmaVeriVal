
Ao inicio do projeto foi definido um roadmap que consistia em 5 sprints cada uma com suas atividades, as atividades mais importantes que seria avaliarmos e propormos mudanças para o projeto enturma, a primeira parte do projeto consistia em colocar o usuário para utilizar o enturma e depois aplicar um questionário, para realizarmos uma coleta de dados com a intenção de descobrimos quais áreas do enturma tinham uma pior avaliação de usabilidade por parte dos usuários.

	Participaram da pesquisa um total de 37 pessoas após a coleta e análise dados foi descoberto que a área que recebeu a pior avaliação por parte dos usuários foi a de comparar turma, Sendo a Selecionar turma a funcionalidade com mais problemas seguida de limpar tela e visualizar dados obtidos.

	A partir da identificação das áreas problematicas fizemos uma priorização de qual área seria mais importante modificar, por conta do tempo que foi previsto houve uma mudança de escopo não havendo mais modificações no código fonte da aplicação, e sim a criação de um prótotipo de média e alta fidelidade, com a construção do prótotipo de média fidelidade foram reaplicados os questionários e feita uma nova coleta de dados, e as propostas que foram sugeridas tiveram uma boa aprovação dos usuários, levando a contrução e validação de um protótipo de alta fidelidade, que gerou uma safisfação maior por parte dos usuários.

uma das lições aprendidas ao longo do desenvolvimento do trabalho, foi principalmente aprender a lidar com o usuário final.

	A primeira dificuldade encontrada foi encontrar pessoas suficientes para realizar a pesquisa, pois muitos disseram que não tinham tempo para realizar o questionário, o que impactou o projeto principalmente na segunda parte, pois não conseguimos realizar a quantidade de entrevistas que desejavamos para a avaliação final.

		um dos maiores aprendizados foi o conhecimento adquirido para a criação de um plano de teste, pois não havia conhecimento para desenvolver o documento que foi de extrema importancia para nos guiar ao longo do desenvolvimento do trabalho e nos auxiliou no cumprimento das atividades planejadas.



	