

O estudo sobre Usabilidade em Interfaces de Sistemas de Software ganhou força quando se percebeu que a utilização de um sistema pode ser muito desagradável, principalmente para leigos em informática ou fora do contexto da aplicação. A experiência de uso de um usuário é extremamente importante para o sucesso do sistema, dessa forma, surgiu a necessidade da utilização de Avaliações de Usabilidade para garantir facilidade de uso para o usuário.

Segundo \cite{avaliacao_heuristica} A primeira norma que definiu o conceito de usabilidade foi a ISO/IEC 9126, de 1991, sobre qualidade de software, que considera a usabilidade como: \textit{Um conjunto de atributos de software relacionado ao esforço necessário para seu uso e para o julgamento individual de tal uso por determinado conjunto de usuários}.

Uma grande referência para o estudo da Usabilidade, é o livro \cite{usabilidade_web}, onde Nielsin especifica 5 parâmetros para a avaliação da Usabilidade. São eles: 

\begin{enumerate}
  \item \textbf{Fácil de Aprender} - Usuário consegue interagir rapidamente com o sistema
  \item \textbf{Eficiente Para Usar} - Uma vez aprendido o funcionamento do sistema, o usuário consegue localizar a informação desejada.
  \item \textbf{Fácil de Lembrar} - O aprendizado do funcionamento não precisa ser feito novamente a cada interação com o sistema, mesmo para usuários ocasionais.
  \item \textbf{Pouco Sujeito a Erros} - Os usuários não têm perigo de cometer erros graves durante a utilização do
sistema e têm a possibilidade de desfazer os que cometem.
  \item \textbf{Agradável de Usar} - Os usuários gostam de interagir com o sistema e se sentem satisfeitos com ele.
\end{enumerate}

Estes 5 (cinco) critérios viabilizam uma Avaliação de Usabilidade bastante simples e eficiente, dessa forma, estes critérios serão muito utilizados ao longo deste trabalho. Para aplicação destes 5 (cinco) critérios, existem inúmeras técnicas na literatura, \cite{avaliacao_usa} classifica a Avaliação da Usabilidade em 3 (três) grandes grupos:
\\

\textbf{- Método de Testes com Usuários:}
	\\

	Envolve a participação direta do Usuário, utilizando entrevistas ou questionários para obter opiniões sobre a experiência de uso do Usuário, e/ou observação de uso, onde o avaliador observa usuários que não conhecem o sistema utilizando o mesmo para recuperar informações sobre tempo, erros e etc. \\

\textbf{- Métodos Baseados em Modelos:}
	\\

	Utilização de Modelos da Inferface do sistema para obter informação sobre a facilidade de uso do Usuário, a Avaliação pode ser gravada para que o avaliador possa analisar o processo novamente. \\

\textbf{- Métodos de Inspeção:}
	\\

	Utiliza as Heurísticas de Nielsen, \cite{usabilidade_interfaces} para avaliação da Usabilidade. O avaliador é um profissional especialista em Usabilidade. \\

	Durante o desenvolvimento deste trabalho, serão trabalhados os 3 (três) grandes grupos de Métodos de Avaliação da Usabilidade para corrigir o máximo de problemas encontrados no Enturma. Nielsen, no livro \cite{usabilidade_web} classifica 10 (dez) Heurísticas sobre a Usabilidade de Sistemas de Software. A seguir, estão apresentadas as \textit{Heurísticas de Nielsen}:

\begin{enumerate}
  \item Visibilidade do estado do sistema;
  \item Correspondência entre o sistema e o mundo real;
  \item Liberdade e controle por parte do usuário;
  \item Consistência e padrões;
  \item Prevenção de erros;
  \item Reconhecimento preferível à memorização;
  \item Flexibilidade e eficiência de uso;
  \item Design estético e minimalista;
  \item Ajuda aos usuários para reconhecer, diagnosticar e se recuperar dos erros;
  \item Suporte e documentação.
\end{enumerate}

	As \textit{Heurísticas de Nielsen} serão usadas para o desenvolvimento de questionários, entrevistas, critérios de avaliação e apoiará a correção de defeitos de Usabilidade do sistema Enturma.

  Com relação aos defeitos de Acessibilidade, a avaliação será com apoio do software \textit{ASES} \cite{avaliacao_ases}, que foi desenvolvido pelo Governo Federal brasileiro e faz uma análise do nível de Acessibilidade do software, utilizando como base o \textit{e-Mag} \cite{e_mag}, que consiste em um conjunto de recomendações a serem consideradas para que o processo de acessibilidade dos sítios e portais do governo brasileiro seja conduzido de forma padronizada e de fácil implementação.

  Segundo \cite{e_mag}, o \textit{e-MAG} foi formulado para orientar profissionais que tenham contato com publicação de informações ou serviços na Internet a desenvolver, alterar e/ou adequar páginas, sítios e portais, tornando-os acessíveis ao maior número de pessoas possível.