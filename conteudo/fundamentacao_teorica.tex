
[Neste capítulo, cabe apresentar o que já foi escrito sobre o tema para embasar teoricamente o presente projeto (já foram realizados trabalhos semelhantes? Qual o ponto de partida? Quais são os autores que serão consultados?). Deverão ser consideradas no mínimo 6 referências bibliográficas entre livros e artigos.]

O estudo sobre Usabilidade em Interfaces de Sistemas de Software ganhou força quando se percebeu que a utilização de um sistema pode ser muito desagradável, principalmente para leigos em informática ou fora do contexto da aplicação. A experiência de uso de um usuário é extremamente importante para para o sucesso do sistema, dessa forma, surgiu a necessidade da utilização de Avaliações de Usabilidade para garantir facilidade de uso para o usuário.

Segundo \cite{avaliacao_heuristica} A primeira norma que definiu o conceito de usabilidade foi a ISO/IEC 9126, de 1991, sobre qualidade de software, que considera a usabilidade como \textit{um conjunto de atributos de software relacionado ao esforço necessário para seu uso e para o julgamento individual de tal uso por determinado conjunto de usuários}.

Uma grande referência para o estudo da Usabilidade, é o livro \cite{usabilidade_web}, onde Nielsin especifica 5 parâmetros para a avaliação da Usabilidade. São eles: 

\begin{enumerate}
  \item \textbf{Fácil de Aprender} - Usuário consegue interagir rapidamente com o sistema
  \item \textbf{Eficiente Para Usar} - Uma vez aprendido o funcionamento do sistema, o usuário consegue localizar a informação desejada.
  \item \textbf{Fácil de Lembrar} - O aprendizado do funcionamento não precisa ser feito novamente a cada interação com o sistema, mesmo para usuários ocasionais.
  \item \textbf{Pouco Sujeito a Erros} - Os usuários não têm perigo de cometer erros graves durante a utilização do
sistema e têm a possibilidade de desfazer os que cometem.
  \item \textbf{Agradável de Usar} - Os usuários gostam de interagir com o sistema e se sentem satisfeitos com ele.
\end{enumerate}

Estes 5 (cinco) critérios viabilizam uma Avaliação de Usabilidade bastante simples e eficiente, dessa forma, estes critérios serão muito utilizados ao longo deste trabalho. Para aplicação destes 5 (cinco) critérios, existem inúmeras técnicas na literatura, \cite{avaliacao_usa} classifica a Avaliação da Usabilidade em 3 (três) grandes grupos:
\\

\textbf{- Método de Testes com Usuários:}
	\\

	Envolve a participação direta do Usuário, utilizando entrevistas ou questionários para obter opiniões sobre a experiência de uso do Usuário, e/ou observação de uso, onde o avaliador observa usuários que não conhecem o sistema utilizando o mesmo para recuperar informações sobre tempo, erros e etc. \\

\textbf{Métodos Baseados em Modelos:}
	\\

	Utilização de Modelos da Inferface do sistema para obter informação sobre a facilidade de uso do Usuário, a Avaliação pode ser gravada para que o avaliador possa analisar o processo novamente. \\

\textbf{Métodos de Inspeção:}
	\\

	Utiliza as Heurísticas de Nielsin, \cite{usabilidade_interfaces} para avaliação da Usabilidade. O avaliador é um profissional especialista em Usabilidade.